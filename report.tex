\documentclass[11pt,a4paper]{article}
\usepackage[utf8]{inputenc}
\usepackage{amsmath}
\usepackage{amsfonts}
\usepackage{amssymb}
\title{Project Report: Managing Transfer-Based Datacenter Connections}
\author{Daniel Naro, Maria Gabriela Valdes, Victoria Beleuta}
\begin{document}
\maketitle
\section{Introduction}
% to do: will add references at the end

For our project we implemented the integer linear programming model described in the paper "Managing Transfer-Based Datacenter Connections" by Adrian Asensio and Luis Velasco. In order to accommodate spikes in cloud demand, while managing the energy consumption of datacenters, it is necessary for datacenter federations to be created. When building large infrastructures cloud operators, with the help of these federations, are capable of reducing capital expenditures by having a dinamic cloud service that uses under-utilized resources in remote datacenters. These dinamic cloud services the huge energy consumption in remote datacenters are minimized, because servers can be turned on and off as needed. To implement these sort of elastic operations, cloud computing strongly depends on the interconnected datacenters through a network providing huge capacity and having the necessary resources available. The interdatacenter traffic results from workloads being encapsulated in virtual machines and moved on different datacenters. Because the amount of data that needs to be transferred is in a huge volume, these features can be covered with a flexgrid optical network, i.e. resource managers can set up and tear down connections of the necessary bitrate only for the amount of time needed to transfer the data. For each of these connections, the frequency slot, i.e. spectrum width, depends on the bandwidth requested and the used modulation format. Elastic connections were suggested due to the fact that not enough resources might be available at the request time, so the manager can dynamically modify the spectrum allocated (RSSA problem) to ongoing transferences and the requested connection to accomodate all traffic and complete it in the requested time. These elastic connections are possible through programmable bandwidth-variable transponders that can provide optical signals of different characteristics.\\

A different approach was proposed in \cite{1}, uses carrier software defined network (SDN) controller on top of ABNO transforms requests from cloud resource managers into connection requests, which helps with a better understanding of network complexity. This paper further explores carrier SDN functionalities by using scheduling techniques in addition to elastic operations, which as a result increases network operator revenues by accepting more connection requests. The SDN controller computes the routing path and necessary spectrum for each transfer and the decides the elastic operations that need to be done on ongoing transferences to fit the new transfer, while satisfying the completion time for all transfers.\\

The rest of our project report is structured as follows. We present the integer linear programming (ILP) model implemented in section 2. Section 3 and 4 defines two metaheuristic algorithms. The results of our project and a comparison of the results obtained are detailed in section 5. Finally, section 6 concludes the report.

\section{IPL Model}

In order to simplify the RSSA problem, we limited our project to only one new transfer being requested. We have a network topology represented by a graph $G(L,E)$ with a set of locations ($L$) and set of fiber connections ($E$). We are given a subset ($D$) of the datacenters' locations, the characteristics for each fiber connection, i.e. a set $S$ of available spectrum slices, and the capacity and number of flows of the optical transponders at each location. The set $R$ contains the ongoing transferences, and origin ($o_{r}$), destination ($d_{r}$), the remaining amount of data ($v_{r}$) to be transferred, the requested completion time ($c_{r}$), the route ($r_{r}$) and slot currently allocated ($s^{0}_{r}$), the scheduled slot allocation ($s^{1}_{r}$) to be performed at time $t^{1}_{r}$, and the scheduled completion time ($t_{r}$) for each transfer in the tuple $\{o_{r}; d_{r}; v_{r}; c_{r}; r_{r}; s^{0}_{r} ; s^{1}_{r} ; t^{1}_{r} ; t_{r}\}$. The details of the new transfer request are given in the tuple $\{o_{r}; d_{r}; v_{r}; c_{r}\}$. The output of our model is the route allocated ($r_{r}$), scheduled completion time ($t_{r}$), bitrate of the connection ($s^{0}_{r}$), the new spectrum allocation ($w^{0}_{r}$), scheduled reallocation ($w^{1}_{r}$, $t^{1}_{r}$), and completion time ($t_{r}$) for each transference request that is rescheduled. Our objective is to minimize the number of connections to be rescheduled to make room for the incoming request.\\

A rescheduling for any transference $r$ needs two spectrum allocations: $s^{0}_{r}$ from $t_{0}$ to $t^{1}_{r}$ and $s^{1}_{r}$ from $t^{1}_{r}$ to $t_{r}$, so that the combined area of the two rectangles allows conveying the remaining amount of data $v_{r}$ and $t_{r} \leq  c_{r}$. To decide dimensions, spectrum and time for the ongoing transferences might lead to nonlinear constraints. Because of this we have written a Java program to precompute any feasible combination of rectangles $A_{0}$ and $A_{1}$ for these transfers, with discretizing time into intervals. A set of precomputed rectangles A for the incoming transfer request is also computed, taking into account completion time and amout of data to be transferred.\\
In order to present the ILP model, we define the necessary parameters and variables:\\\\
\begin{table}
\small
\begin{tabular}{c l}
$E$ & set of fiber links in the network, index $e$.\\
$S$ &  set of frequency slices, index $s$.\\
$T$ & set of time intervals, index $t$.\\
$R$ & set of ongoing transferences, index $r$.\\
$P$ & set of routes between origin and destination for the new request, index $p$.\\
$A$ & set of candidate areas for the requested trans- ference, index $a$.\\
$A^{0}_{r}$ & set of candidate areas for $r$ between $t_{0}$ and $t^{1}_{r}$ and a spectrum allocation.\\
$A^{1}_{r}$ & set of candidate areas for $r$ between $t_{1}$ and $t_{r}$ and a spectrum allocation.\\
$\omega_{ar}$ & 1 if transference r was assigned to area $a$, 0 otherwise.\\
$\delta_{as}$ & 1 if area a includes slice $s$, 0 otherwise.\\
$\gamma_{at}$ & 1 if area $a$ includes time interval $t$, 0 otherwise.\\ 
$\rho_{pe}$ & 1 if route $p$ uses link $e$, 0 otherwise.\\
$\rho_{re}$ & 1 if transference $r$ uses link $e$, 0 otherwise.\\
$\beta_{raa'}$ & 1 if pair of areas $a \in A_{0}$ and $a' \in A_{1}$ for transference $r$ is feasible.\\
$\delta_{kest}$ & 1 if fixed-bitrate connection $k$ uses slice $s$ in link $e$ at time $t$, 0 otherwise.\\
$\alpha_{est}$ & 0 if slice $s$ in edge $e$ is used by any fixed-bitrate connection time interval $t$.\\
\end{tabular}
\end{table}
\newline
Variables:\\
\begin{table}
\small
\begin{tabular}{c l}
$x_{ap}$ & 1 if the new transference is assigned to area $a$ through route $p$.\\
$y_{r}$ & 1 if transference $r$ is rescheduled, 0 otherwise.\\
$x^{0}_{ar}$ & 1 if transfer $r$ is assigned to area $a \in A_{0}$.\\
$x^{1}_{ar}$ & 1 if transfer $r$ is assigned to area $a \in A_{1}$.\\
\end{tabular}
\end{table}
\newline
And the ILP model is as follows:\\\\
$minimize \sum_{r\in R} y_{r} $\\\\
subject to:\\\\
$\sum_{a \in A} \sum_{p \in P}$ $x_{ap} = 1$\\\\
$\sum_{a \in A_{0(r)}}$ $x^{0}_{ar}=1$    $\forall r \in R$\\\\
$\sum_{a \in A_{1(r)}}$ $x^{1}_{ar}=1$    $\forall r \in R$\\\\
$\sum_{a' \in A_{1(r)}}$ $\beta_{raa'}x^{1}_{a'r} \geq x^{0}_{ar}$     $\forall r \in R,$ $a \in A_{0(r)}$\\\\
$x^{0}_{ar} - \omega_{ar} \leq y_{r}$     $\forall$ $r \in R,$ $a \in A_{0(r)}$\\\\
$\sum_{r \in R} \sum_{a \in A_{0(r)}}$ $\delta_{as} * \gamma_{at} * \rho_{re} * x^{0}_{ar} + \sum_{r \in R} \sum_{a \in A_{1(r)}} \delta_{as} * \gamma_{at} * \rho_{re} * x^{1}_{ar}$\\ 
$+ \sum_{a \in a} \sum_{p \in P} \delta_{as} * \gamma_{at} * \rho_{pe} * x_{ap} \leq \delta_{est}$    $\forall$ $e \in E,$ $s \in S,$ $t \in {\{t^{0},...,T\}}$\\\\

The first constraint guarantees that the transfer request is served with route $p$ and rectangle $a$. Then we select the feasible rectangles for the ongoing transferences with the next three constraints. After words, with the sixth constraint we decide whether a transference needs to be reallocated or not and the last constraint guarantees that there are no links used by more than one constraint for each time interval. We have implemented and tested our ILP model with the ILOG CPLEX studio.

\section{Metaheuristic 1}
\section{Metaheuristic 2}
\section{Results}
\section{Conclusion}
\end{document}
